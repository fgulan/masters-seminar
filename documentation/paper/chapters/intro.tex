Očitavanje slova je problem koji se rješava već dugi niz godina. Još od 1980-ih godina, početkom ere računala, počeli su se razvijati prvi programi koji bi zamijenili čovjeka u raznim operacijama, pa tako i u očitavanju slova. Primjena je bila razna, od očitavanja brojeva računa u bankama do očitavanja poštanskih brojeva prilikom sortiranja pisama.

Svi ti sustavi su većinom bili namijenjeni za očitavanje tiskanih brojeva. Postojala je nekolicina sustava i za očitavanje rukom pisanih brojeva no oni su zahtijevali točno određene načine pisanja kako bi mogli ispravno raditi.

Razvojem računala i tehnika strojnog učenja, navedeni su poslužili prilikom izgradnje sustava za očitavanje znakova. Tako danas imamo raznorazne komercijalne alate koji bez većih problema mogu vrlo brzo i efikasno očitati cijele stranice tiskanoga tekstualnoga dokumenta.

No, očitavanjem rukom pisanih slova danas još uvijek nije unaprijeđeno na razinu točnosti kojom ljudi prepoznaju pojedina slova. Pokazalo se da najbolje rezultate prilikom prepoznavanje rukom pisanih slova ostvaruju sustavi izgrađenih na raznoraznim modelima strojnog učenja.

Kod pristupa gdje se koriste modeli strojnog učenja dolazi do nekoliko ključnih problema. Prvi od njih je skup za učenje koji treba biti skupljen od velikog broja ljudi kako bi izgrađeni sustav bio što bolje otporan na različite rukopise. Drugi problem je veličina samog skupa za učenje, jer što je skup veći, to je klasifikatora računski zahtjevnije za naučiti. Također, kod navedenoga pristupa postoji problem što klasifikator naučen za jedno pismo, neće raditi ispravno za drugo pismo, na primjer sustav za očitavanje latiničnih znakova neće raditi ispravno prilikom očitavanja kineskih slova.

Kroz ovaj rad prikazat će se izgradnja takvog sustava kroz prizmu dubokog učenja i konvolucijskih neuronskih mreža.

