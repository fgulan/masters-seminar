Tehnika očitavanja rukom pisanih slova se razvija još od polovice prošlog stoljeća i od tada je u mnogočemu evoluirala. U ovom radu, kao klasifikator pojedinog slova, korištena je konvolucijska neuronska mreža.

U radu je predstavljen cjeloviti postupak prikupljanja skupa podataka i obrade slike kako bi se ulazna slika pripremila za svrhu učenja konvolucijske neuronske mreže. Također, opisana je arhitektura korištene neuronske mreže i postupak učenja iste.

Dobiveni rezultati su vrlo zadovoljavajući za dani skup podataka. Usporedbe radi, u radu \citep{zavrsni} prikazana je obična neuronska mreža učena na nešto manjem skupu podataka uz ručno izlučivanje značajki, te je dobivena točnost klasifikatora za hrvatsku abecedu iznosila $78.44\%$.

Konvolucijske neuronske mreže su se pokazale kao iznimno dobar klasifikator u području klasificiranja slike. U budućnosti bi se mogao proširiti sam skup podataka te bi se kao klasifikator mogle isprobati nešto kompliciranije konvolucijske neuronske mreže učene naprednijim algoritmima učenja, poput one navedene u radu \citep{lenet5hecr}.
